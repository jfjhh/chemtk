\documentclass[final,letterpaper,12pt]{article}
\usepackage[english]{babel}
\usepackage[utf8]{inputenc}
\usepackage[T1]{fontenc}
\usepackage{listings}
\usepackage{tikz}
\usepackage{tikz-qtree}

\renewcommand*\ttdefault{pcr}

\author{Alex Striff}
\title{ChemTK Command Specification \\ \large
Command Tree Structure, BNF, and YAML Data File Specification}
\date \today

\begin{document}

\maketitle
\clearpage

\section{Example Tree}

The tree of the command, \texttt{'ei n'}, is shown below.
In English, this command says,
\textit{print the element information about nitrogen}.

\Tree [
	[
		[ [ Element ].\texttt{e} [
			[ Information ].\texttt{i}
		].Token ].Token
		[ {Optional Space} ].\texttt{' '}
		[ [ Nitrogen ].\texttt{n} ].Value
	].\texttt{ei n}
].Command

\section{BNF Specification}

\lstset{numbers=left, numberstyle=\small}
\texttt{\lstinputlisting{commands.bnf.txt}}

\section{The Global Command Tree}

The command tree will look something like this. The struct member \texttt{name}
was ommitted for space.

\Tree [
	[ \texttt{NUL} ].T
	[
		[ \texttt{'e'} ].T
		[
			[ \texttt{'i'} ].T
			[ NULL ].N
			[ \texttt{foo()} ].F
		].N
		[ \texttt{NULL} ].F
	].N
	[
		[ \texttt{'c'} ].T
		[ NULL ].N
		[ \texttt{bar()} ].F
	].N
	[ \texttt{NULL} ].F
].N

\section{Current Commands YAML File}

\lstset{numbers=left, numberstyle=\small}
\texttt{\lstinputlisting{../../data/commands.yml}}

\end{document}

